\documentclass[a4paper,12pt]{article}

\usepackage[utf8]{inputenc}
\usepackage[T1]{fontenc}
\usepackage{lmodern}
\usepackage{a4wide}
\usepackage{url}
\usepackage{color}
\usepackage[swedish]{babel}


\newcommand{\todo}[1]{{\color{red} #1}}


\begin{document}
\title{5DV160/HT17: OU3---Tabeller}
\author{\emph{\color{red}Förnamn Efternamn}\\{\color{red}\url{din-email@cs.umu.se}}}
\date{\emph{\color{red}Aktuellt datum}}
\maketitle
\tableofcontents
\thispagestyle{empty}
\newpage




\section{Inledning}
\label{sec:inledning}

\begin{itemize}
\item \todo{Ge en inledning så att läsaren förstår vad som kommer.}
  \begin{itemize}
  \item \todo{Ge en översiktlig beskrivning av datatypen Tabell.}
  \item \todo{Tala om att du implementerat två varianter och fått en tredje. Hänvisa framåt till Avsnitt~\ref{sec:implementation}.}
  \item \todo{Nämn att du testat implementationernas prestanda. Hänvisa framåt till Avsnitt~\ref{sec:experiment}.}
  \item \todo{Nämn att du analyserat och jämfört implementationernas prestanda. Hänvisa framåt till Avsnitt~\ref{sec:analys}.}
  \end{itemize}
\end{itemize}


\section{Implementation}
\label{sec:implementation}

\begin{itemize}
\item \todo{Beskriv med text och bilder de tre olika implementationerna (dvs även den du fått given).}
\item \todo{Följande frågor bör bli besvarade av text och/eller bild:}
  \begin{itemize}
  \item \todo{Hur hanteras försök till borttagning av ej existerande nyckel?}
  \item \todo{Hur hanteras dubletter?}
  \end{itemize}
\end{itemize}


\subsection{Tabell som enkellänkad lista}
\label{sec:tabell-som-enkell}

\begin{itemize}
\item \todo{Beskriv den givna implementationen.}
\end{itemize}


\subsection{Tabell som array}
\label{sec:tabell-som-array}

\begin{itemize}
\item \todo{Beskriv implementationen som är baserad på en array.}
\end{itemize}


\subsection{Tabell som hashtabell}
\label{sec:tabell-som-hasht}

\begin{itemize}
\item \todo{Beskriv implementationen som är baserad på en hashtabell.}
\end{itemize}



\section{Experiment}
\label{sec:experiment}

\begin{itemize}
\item \todo{Beskriv resultatet av tidsmätningarna.}
\item \todo{Det ska framgå vad som testats, hur det testats, och vilka parametrar som använts.}
\item \todo{Sammanfatta resultaten i en tabell (tänk på antalet signifikanta siffror).}
\end{itemize}



\section{Analys}
\label{sec:analys}

\begin{itemize}
\item \todo{Analysera resultaten från Avsnitt~\ref{sec:experiment}.}
  \begin{itemize}
  \item \todo{Försök förklara de observerade skillnaderna/likheterna mellan implementationerna för respektive experiment.}
  \end{itemize}
\end{itemize}


\section{Slutsats}
\label{sec:slutsats}

\begin{itemize}
\item \todo{Ge en kort sammanfattning av rapporten.}
\item \todo{Dra en slutsats: när du väger samman resultaten från samtliga experiment, vilken implementation tycks då vara snabbast?}
\item \todo{Försök ge en förklaring till varför den implementationen är snabbast.}
\end{itemize}



\section*{Referenser}
\label{sec:referenser}

\begin{itemize}
\item \todo{Referenslista}
\item \todo{Samtliga ska vara refererade i texten}
\end{itemize}







\end{document}
